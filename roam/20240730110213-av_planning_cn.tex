% Created 2024-07-30 Tue 11:07
% Intended LaTeX compiler: xelatex
\documentclass[11pt]{article}
\usepackage{graphicx}
\usepackage{longtable}
\usepackage{wrapfig}
\usepackage{rotating}
\usepackage[normalem]{ulem}
\usepackage{capt-of}
\usepackage{hyperref}
\author{Binjian Xin}
\date{\today}
\title{AV-Planning\textsubscript{CN}}
\hypersetup{
 pdfauthor={Binjian Xin},
 pdftitle={AV-Planning\textsubscript{CN}},
 pdfkeywords={},
 pdfsubject={},
 pdfcreator={Emacs 30.0.60 (Org mode 9.8)}, 
 pdflang={English}}
\usepackage{biblatex}

\begin{document}

\maketitle
\section*{多模态(多目标行为和轨迹预测)}
\label{sec:orge4fac1d}
\subsection*{区分认识论与偶然论的多模态(epistemic vs aleatoric):前者可通过更新测量消除,后者无法消除}
\label{sec:org15b9d10}
\subsection*{表示应该是多模态的:比如非端到端采用混合高斯模型}
\label{sec:org00b743d}
\subsection*{端到端从数据中学习多模态}
\label{sec:org35f170f}
\subsubsection*{潜在模型(CVAE)}
\label{sec:orgc62f135}
\subsubsection*{扩散模型(Diffusion)}
\label{sec:org796b078}
\subsection*{保持多模态分布并从感知中实时更新,随时间演变成最终几个最终无法消除的多模态或单一模态}
\label{sec:org1e6cb4b}
\section*{特定驾驶风格自适应匹配}
\label{sec:org1689ca5}
\subsection*{DAgger,简单,实用,有效}
\label{sec:orgba4630b}
\subsection*{RLHF(使用Bradley-Terry模型的奖励模型[比较],在线奖励模型与在线HF,state(运动轨迹),action)}
\label{sec:orgafa9095}
\subsubsection*{使用回归loss,可切换至DPO方法,RL->常规MLE,稳定,高效训练}
\label{sec:org609c0cc}
\subsubsection*{具体奖励:行为策略与目标策略之间的KL散度}
\label{sec:org6215884}
\subsubsection*{抽象奖励:最小化观察分布漂移->修复数据分布漂移}
\label{sec:orgf720cd3}
\subsection*{基于RL的扩散策略}
\label{sec:orgeb02d7d}
\subsubsection*{使用ControlNet}
\label{sec:org2bedaf9}
\section*{扩散模型为何高效?}
\label{sec:org70295d5}
\subsection*{多模态概率分布拟合的SOTA:通过从数据分布中采样数据来近似任意复杂(包括多模态)分布}
\label{sec:orge5a7566}
\subsection*{通过得分函数(Score Function)把原始随机分布分布迁移到数据样本分布(真实分布)}
\label{sec:org314ff9e}
\subsection*{如果数据分布漂移,通过从最新数据中采样来适应学习模型到新的数据分布}
\label{sec:org3e5db83}
\subsubsection*{通过强化学习训练扩散模型}
\label{sec:org506432e}
\subsection*{训练遵循与常规深度学习相同的方法}
\label{sec:orgf517f6a}
\subsection*{数据分布拟方法比较}
\label{sec:org024d9ce}
\subsubsection*{GAN:由于博弈不稳定}
\label{sec:orgefa2c83}
\subsubsection*{VAE:多模态是超参数Z(潜在维度),需要调试}
\label{sec:org9174a8a}
\subsubsection*{流模型:效率低下}
\label{sec:org810750d}
\section*{隐空间(内嵌)}
\label{sec:org5c87287}
\subsection*{用于端到端训练的内嵌层始终优于预训练的tokens}
\label{sec:org4d69358}
\section*{基于学习的方法(端到端)}
\label{sec:orgf0258c3}
\subsection*{生成/收集样本 --> 通过采样改进:采样/数据分布来近似真实统计数据}
\label{sec:orgf2b3610}
\subsubsection*{期望}
\label{sec:org34ddc56}
\subsubsection*{梯度}
\label{sec:orgf5a3e2c}
\subsubsection*{分布}
\label{sec:orgba88e9d}
\subsection*{用RL解决开放性问题}
\label{sec:orgd675e27}
\end{document}
