\documentclass[tikz,border=3mm]{standalone}
\usetikzlibrary{positioning,fit}
\begin{document}

\tikzset{%
	neuron missing/.style={
			draw=none,
			scale=2,
			text height=0.333cm,
			execute at begin node=\color{black}$\vdots$
		},
}

% The command \DrawNeuronalNetwork has a list as argument, each entry is a
% layer. each entry has the form
%  Layer name/number of nodes/color/missing node/label/symbolic number
% where
% * layer name is, well,  the name of the layer
% * number of nodes is the number of neurons in that layer (including the missing neuron)
% * color is the color of the layer
% * missing node denotes the index of the missing neuron
% * label denotes the label of the layer
% * symbolic number denotes the symbol that indicates how many neurons there are
% * node content
\newcommand{\DrawNeuronalNetwork}[2][]{
	\xdef\Xmax{0}
	\foreach \Layer/\X/\Col/\Miss/\Lab/\Count/\Content [count=\Y] in {#2}
		{\pgfmathsetmacro{\Xmax}{max(\X,\Xmax)}
			\xdef\Xmax{\Xmax}
			\xdef\Ymax{\Y}
		}
	\foreach \Layer/\X/\Col/\Miss/\Lab/\Count/\Content [count=\Y] in {#2}
		{\node[anchor=south] at ({2*\Y},{\Xmax/2+0.1}) {\Layer};
			\foreach \m in {1,...,\X}
				{
					\ifnum\m=\Miss
						\node [neuron missing] (neuron-\Y-\m) at ({2*\Y},{\X/2-\m}) {};
					\else
						\ifnum\m=\X
							\node [circle,fill=\Col!50,minimum size=1cm] (neuron-\Y-\m) at ({2*\Y},{\X/2-\m}) {\Content};
						\else
							\node [circle,fill=\Col!50,minimum size=1cm] (neuron-\Y-\m) at ({2*\Y},{\X/2-\m}) {\m};
						\fi
						\ifnum\Y=1
						\else
							\pgfmathtruncatemacro{\LastY}{\Y-1}
							\foreach \Z in {1,...,\LastX}
								{
									\ifnum\Z=\LastMiss
									\else
										\draw[->] (neuron-\LastY-\Z) -- (neuron-\Y-\m);
									\fi
								}
						\fi
					\fi
					\ifnum\Y=1
						\ifnum\m=\X
							\draw [overlay] (neuron-\Y-\m) -- (state);
						\else
							\ifnum\m=\Miss
							\else
								\draw [overlay] (neuron-\Y-\m) -- (state);
							\fi
						\fi
					\else
					\fi
				}
			\xdef\LastMiss{\Miss}
			\xdef\LastX{\X}
		}
}
\begin{tikzpicture}[x=1.5cm, y=1.5cm, >=stealth,font=\sffamily,nodes={align=center}]
	\begin{scope}[local bounding box=T]
		\path  node[draw,minimum width=6em,minimum height=4em] (state) {State};
		\begin{scope}[local bounding box=NN]
			\DrawNeuronalNetwork{Input Layer/5/green/4/inp//90,
				Hidden Layer 1/5/blue/4/h1//256,
				Hidden Layer 2/5/cyan/4/h2//256,
				Output Layer/4/red/3/out//68}
		\end{scope}
		\path (NN.south) node[below]{Estimated parameter\\ $\theta$};
		\path (NN.east) -- node[above]{Policy\\ $\Pi(\theta,a)$}++ (4em,0);
	\end{scope}
	\node[fit=(T),label={[anchor=north west]north west:Actor},inner sep=1em,draw]
	(TF){};
	\node[below=3em of TF,draw,inner sep=1em] (Env) {Environment};
	\draw[<-] (TF.200) -- ++ (-1em,0) |- (Env.160) node[pos=0.45,right]{$r_t$};
	\draw[<-] (TF.180) -- ++ (-2em,0) |- (Env.180) node[pos=0.45,left]{$s_t$};
	\draw[->] (NN.east) -- ++ (7em,0)node[right]{$a_t$} |- (Env);
\end{tikzpicture}
\end{document}
